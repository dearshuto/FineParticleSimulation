 \section{序論}
 個体の粒子が無数に集合したものを粉体とよぶ.
 物資の輸送や貯蔵, 薬品の製造など, 粉体は日常生活に欠かせないものである.
 この粉体の性質を定式化し, 工業に応用しようとする学問が粉体工学である.\\
 粉体工学の分野では, 粉体に対してさまざまな研究が行われているが, その1つが粉体の粒子径に注目した研究である.
 構成粒子の大きさによって, 粉体の性質は大きく異なるということが粉体工学の分野では知られている.
 そこで, 粉体の粒子径とその粉体がもつ性質によって粉体を4つグループに分類したのがGeldartらによって提唱されたGeldart Map\cite{geldart_map}である.\\
 ちなみに, 本来Geldart Mapは流動化(fluidization)のしやすさを計るための分類である.
 流動化とは粉体を輸送するための工業的手法の1つである.
 粉体に空気を大量に送り込み, 粉体と空気を混合することで粉体を流動させるという方法である.\\
 さて, コンピュータグラフィックスの分野でも粉体に関する研究は多く存在する.
 しかし, Geldart Mapに従うと, 既存の研究で実現されているのはGeldart Bグループの粉体であり, 
 さらに小さな粒子からなるGeldart Cグループの粉体に関する研究はまだない(表\ref{tab:geldart_map}).\\
 
  \begin{table}[h]
   \begin{center}  
    \caption{Geldart Map}
    \begin{tabular}{|c|c|l|l|} \hline
     名前 & 粒子径   &  性質              & 例 \\ \hline
     D & 500μm〜 &   空間充填率が低い        & コーヒー豆 \\
     B & 〜500μm & 流動性が高い        & 砂 \\
     A & 〜200μm & 流動しにくい   & 触媒 \\ 
     C & 〜100μm & 付着性がある         & 小麦粉 \\
     \hline
    \end{tabular}
    \label{tab:geldart_map}
   \end{center}
  \end{table}
  
 Geldart Cグループに分類されるような粉体の特徴として, いちじるしく流動性が損なわれていることがあげられる.
 これは粉体の単位体積あたりの表面積が大きく, 粒子間の相互作用がひじょうに大きいことに由来する.
 また, Geldart AグループとGeldartCには数値として判断できるような違いはなく, 実験によって現象を観察することで経験的に判別される.
Geldartの論文でも, Geldart Cグループの粉体については``Powders which are in any way cohesive belong in this category.''という言及の仕方をとっている.
 これに対し, Geldadrt DとGeldart Bは粉体の物理属性を測定するだけで分類可能である\\
 本研究では粒子間の相互作用に注目し, Geldart Cに分類されるような粉体の現象を再現することを目的とする.
 
