%%%%%%%%%%%%%%%%%%%%%%%%%%%%%%%%%%%%%%%%%%%%%%%%%%%%%%%%%%%%%%%%
%%%%%%%  Example: extended abstract for master thesis
%%%%%%%  version 1.0
%%%%%%%  file name: template.tex
%%%%%%%%%%%%%%%%%%%%%%%%%%%%%%%%%%%%%%%%%%%%%%%%%%%%%%%%%%%%%%%%
%--------------- start preamble -------------------------------
\documentclass[a4paper, 10pt]{jarticle} % 10pt fonts, default fonts
%\documentclass[a4paper,11pt]{jarticle} % 11pt fonts
%\documentclass[a4paper,12pt]{jarticle} % 12pt fonts
%--------------------------------------------------------------
\usepackage{masterabs} % 修士論文アブストラクトのスタイルファイル
%--------------------------------------------------------------
%\usepackage{amsmath,amsthm,mathrsfs} % amslatex モードの指定
%\usepackage{amsfonts,amssymb,txfonts} % amsfonts の指定
\usepackage[dvipdfmx]{graphicx} % 図の挿入の指定 (\includegraphicsなど)
\usepackage{subfigure}%画像を横に並べる
\usepackage{lettrine}%書き出しをおしゃれに
\usepackage{amsmath}%数式を書くための便利機能
\usepackage{nidanfloat}%図や表のぶち抜き表示を可能にする
%--------------------------------------------------------------
% \columnseprule = 0.4pt % two columnの真ん中に縦線を引く
%--------   英文の場合: 表,図、参考文献を英語に変更 ----------------
% \initenglish % 本文が英文の場合は % を取る(表=>Tab., 図=>Fig.など)
%--------------------------------------------------------------------
%


%-------------- end preamble ----------------------------------
%
%%%%%%    TEXT START    %%%%%%

\begin{document}
%
%-------------- two column -------------------------
 \twocolumn[ % two column の場合は,先頭の % を取る
%---------------------------------------------------
%
%---------------------------------------------------------------------------
% \no_tlfnmark % タイトルの最後にfootnote markを付けない場合は,先頭の % を取る
%---------------------------------------------------------------------------
%%% タイトルが 1行 \title{タイトル}を使う
%%% タイトルが 2 行にわたるときは \2ltitle{1行目}{2行目}を使う
%---------------------------------------------------------------
%\title{粉体工学資料} % 1 行用
%
\2ltitle{粉体工学資料}{クラックとチャネリングの再現} % 2 行用
%
%-------------------------------------------
% 日本語指導教員,著者名など
%-------------------------------------------
\begin{preliminary}
\profname{藤代 一成}       %% 指導教員の名前 + 講師,准教授,教授
\name{81519945}{鹿間 脩斗} %% 学籍番号, 著者名
\end{preliminary}
%
%---------- two column ----------------------
 ]% two columnの場合は,先頭の % を取る
%--------------------------------------------
%
%------- footnote に英文のタイトルを記述したいとき ----------------
\etitle{Particuology Document}
%----------------------------------------------------------------
%
\init_fnmark % 脚注マークの初期化(アラビア数字に変更)
%%%%%%%%%%%%%%%%%%%%%%%%%%%%  本文 %%%%%%%%%%%%%%%%%%%%%%%%%%%%%

 \section{序論}
 個体の粒子が無数に集合したものを粉体とよぶ.
 物資の輸送や貯蔵, 薬品の製造など, 粉体は日常生活に欠かせないものである.
 この粉体の性質を定式化し, 工業に応用しようとする学問が粉体工学である.\\
 粉体工学の分野では, 粉体に対してさまざまな研究が行われているが, その1つが粉体の粒子径に注目した研究である.
 構成粒子の大きさによって, 粉体の性質は大きく異なるということが粉体工学の分野では知られている.
 そこで, 粉体の粒子径とその粉体がもつ性質によって粉体を4つグループに分類したのがGeldartらによって提唱されたGeldart Map\cite{geldart_map}である.\\
 ちなみに, 本来Geldart Mapは流動化(fluidization)のしやすさを計るための分類である.
 流動化とは粉体を輸送するための工業的手法の1つである.
 粉体に空気を大量に送り込み, 粉体と空気を混合することで粉体を流動させるという方法である.\\
 さて, コンピュータグラフィックスの分野でも粉体に関する研究は多く存在する.
 しかし, Geldart Mapに従うと, 既存の研究で実現されているのはGeldart Bグループの粉体であり, 
 さらに小さな粒子からなるGeldart Cグループの粉体に関する研究はまだない(表\ref{tab:geldart_map}).\\
 
  \begin{table}[h]
   \begin{center}  
    \caption{Geldart Map}
    \begin{tabular}{|c|c|l|l|} \hline
     名前 & 粒子径   &  性質              & 例 \\ \hline
     D & 500μm〜 &   空間充填率が低い        & コーヒー豆 \\
     B & 〜500μm & 流動性が高い        & 砂 \\
     A & 〜200μm & 流動しにくい   & 触媒 \\ 
     C & 〜100μm & 付着性がある         & 小麦粉 \\
     \hline
    \end{tabular}
    \label{tab:geldart_map}
   \end{center}
  \end{table}
  
 Geldart Cグループに分類されるような粉体の特徴として, いちじるしく流動性が損なわれていることがあげられる.
 これは粉体の単位体積あたりの表面積が大きく, 粒子間の相互作用がひじょうに大きいことに由来する.
 また, Geldart AグループとGeldartCには数値として判断できるような違いはなく, 実験によって現象を観察することで経験的に判別される.
Geldartの論文でも, Geldart Cグループの粉体については``Powders which are in any way cohesive belong in this category.''という言及の仕方をとっている.
 これに対し, Geldadrt DとGeldart Bは粉体の物理属性を測定するだけで分類可能である\\
 本研究では粒子間の相互作用に注目し, Geldart Cに分類されるような粉体の現象を再現することを目的とする.
 


\section{粉体の基本原理}
粉体の現象は, 流体とは異なり, 各粒子が実際に衝突を起こすことによって引き起こされている.
そこで, 粉体層を離散化するさいには, 剛体としての性質が重要な要素となってくる.
すなわち, 粒子の形状や回転の取扱いである.
これらはモデル化において, そして流体シミュレーションとの明確な差別化として, ひじょうに重要な位置付けにある.


 \section{基礎式}
  並進運動と回転運動を計算することで各粒子の動きを算出する.
  並進運動は式\eqref{eq:base_velocity}のようにニュートンの第2法則として与えられる.

  \begin{eqnarray}
   \label{eq:base_velocity}
   m_{i}{\boldsymbol a}_i &=& {\boldsymbol F}_{sum}\\
   {\boldsymbol F}_{sum} &=& \sum {{\boldsymbol F}_ C }_i + \sum {{\boldsymbol F}_ a }_i + {\boldsymbol F}_{g}\\
  \end{eqnarray}
  ここで, $m_i$, ${\boldsymbol a}_i$, ${{\boldsymbol F}_C}_i$, ${{\boldsymbol F}_a}_i$, ${\boldsymbol F}_g$, はそれぞれ, 粒子$i$の質量, 接触力, ファンデルワールス力による吸着力, 重力である.

%% 粉体の崩壊条件についての記述。どうすべきかなぁ
%  ただし
%  \begin{eqnarray}
%   {\boldsymbol F}_{sum} &= 
%    \left\{
%     \begin{array}{l}
%      {\boldsymbol F}  (崩壊条件を満たしているとき)\\
%      {\boldsymbol 0}  (else)
%     \end{array}
%    \right. \nonumber
%  \end{eqnarray}

  
  また, 回転運動は式\eqref{eq:angular_velocity}のように与えられる.

  \begin{eqnarray}
   \label{eq:angular_velocity}
    \dot{\omega_i} &=& \frac{\sum {\boldsymbol T}_i }{I_i}
  \end{eqnarray}


  ここで, $\omega_i$, ${\boldsymbol T}_i$, $I_i$は, それぞれ, 粒子$i$の質量, 加速度, 角速度, トルク, 回転モーメントである.
 離散化した各粒子に働く力を求め, この式に落とし込んでいくのが基本的な流れでとなる.
  

   \section{粉体の離散化}
 シミュレーションに使用する粉体の形状は粉体工学の世界でも研究の対象となっている.
 広く使用されているのはもっとも計算がシンプルで高速な処理が可能な球体であるが, それ以外の形状を利用したシミュレーションについての研究が存在する.
 鈴木らは穀粒のシミュレーションには楕円体を使用したシミュレーションが効果的であるということを提唱した\cite{elipsoid_collision}.
 Frerasらはキューブを利用したシミュレーションの有用性を提唱した\cite{cubic_collision}.
 これらの研究では, 複雑な形状に対しても球体と同様のアプローチを取るための手法が提案されている.\\
 しかし, グラフィックスの分野では厳密な粒子形状を追求する必要性はない.
 そこで粒子形状は球体を利用するものとする.
  
 \section{接触力の計算}

  \subsection{レオロジーモデル}
  簡単な力学モデルを組み合わせることで複雑な力学現象をモデル化したものがレオロジーモデルであり, 現象論的なアプローチに分類される. \\
  粉体のシミュレーションに広く使用されているのはフォークト(Voigt) モデルである.
  これは粒子間にばねとダッシュポッドが並列に接続していると考える, 粒子の接触力に注目したモデルである.
  粒子間にばねとダッシュポッドが直列に接続していると考えるマックスウェル(Maxwell)モデルもあるが, これは粉体の圧縮現象に注目したモデルである.
  フォークトモデルによってモデル化すると, 粒子間に働く接触力は以下の式で与えられる.

  \begin{flalign}
   \label{eq:base_equation}
   {\boldsymbol F}  &=  {\boldsymbol F}_{spring} + {\boldsymbol F}_{dash-pod}&\\
   {\boldsymbol F}_{spring} &=  -{k}{\boldsymbol \delta}_{ij} &\\
   {\boldsymbol F}_{dash-pod} &=  -\eta{\boldsymbol v}_{C_{ij}}&\\
   \label{eq:voigt}
   {\boldsymbol \delta_{ij} } &: めり込み量 \nonumber \\
   \eta &: 粘性減衰係数 \nonumber \\
   {\boldsymbol {\boldsymbol v} }_{ij} &: 相対速度 \nonumber
  \end{flalign}


  この式を用いて法線方向に働く力と接線方向に働く力を計算する.
  そして, それぞれ並進運動と回転運動に寄与する成分を取り出す.

  \begin{flalign}
   F_n &=  [( {\boldsymbol F}_{spring} + {\boldsymbol F}_{dash-pod}) \cdot {{\boldsymbol r}_i}'] {{\boldsymbol r}_i}'\\
   F_t &= {\boldsymbol r}_i \times ( {\boldsymbol F}_{spring} + {\boldsymbol F}_{dash-pod})
  \end{flalign}
  ここで, ${\boldsymbol r}_i$は接触点の中心から見た相対位置であり, ${\boldsymbol r}_i' = -{\boldsymbol r}_i$である.\\
  粒子形状を球体でモデル化した場合, 接触点の法線は必ず衝突した粒子の中心を結ぶ直線上にある.
  そのため, フォークトモデルによって計算した力を分解する必要はない.

  
   \subsubsection{ばね}
   粒子間の反発や弾性を表現するモデル.
   ばね係数kは経験にもとづいて決定することが多い.
   粒子系に対するオーバーラップが0.1\%〜1.0\%程度にすると粉体の挙動を適切に再現できるという研究結果がある\cite{dem_spring_coefficient}.
   注目する現象(例えば輸送機による大掛かりな粉体の移動)が粒子間の弾力に関係ない場合, その現象を顕著にするために, さらに弱いばね係数を用いることもある.\\
   また, ばね係数は粉体の相互作用を計算するときの力の方向に関係なく, 一定の値を使用して構わないとされている.
   
   \subsubsection{めり込み量}
   ばねが粒子に加える力を計算するには粒子間のめり込み量が必要である.
   $ {\boldsymbol n}_{C_{ij}}$ を接触点における法線とすると, 法線方向のめり込み量は以下の式で与えられる.
   \begin{flalign}
    {\boldsymbol \delta_{t_{ij}} } & = k\delta_{ij_n }{\boldsymbol n}_{C_{ij}}
   \end{flalign}

   接線方向のめり込み量は, 衝突が生じてからの接線方向の移動を足し合せることで与えられる.
   \begin{flalign} 
    {\boldsymbol \delta}^n_{ij} = |{\boldsymbol \delta}^{n-1}_{ij}| {\boldsymbol t}_{ij} + {\boldsymbol v}^n_t \Delta t
   \end{flalign}
   
   ${\boldsymbol t}_{ij}$は接線方向の時間変化に対応するためのベクトルである.
   \begin{flalign} 
    {\boldsymbol t}_{ij} =  \frac{ {\boldsymbol \delta}^{n-1}_t}{| {\boldsymbol \delta}^{n-1}_t|}
   \end{flalign}  
   
   
   \subsubsection{ダッシュポッド}
   個体粒子間の接触および衝突による, エネルギー減衰の現象, すなわち粘性をモデル化したもの.
   粘性減衰係数はばね係数と関連付けられた以下の式で与えられる
   
   \begin{flalign}
    \eta  &= -2\ln e \sqrt{ \frac{mk}{\pi^2 + (\ln e)^2} }\\
    m&:質量 \nonumber\\
    k&:ばね係数 \nonumber\\
    e&: 反発係数 \nonumber
   \end{flalign}
   
   質量が異なる粒子による衝突では以下の式で定義される換算質量を使用する.
   \begin{flalign}
    m = \frac{m_i m_j}{m_i + m_j}
   \end{flalign}
   
   粉体の内部はほとんど動かないので, 1つの大きな粒子であると考えても問題ない.
   将来的には内部と外部とで大きさが違う粒子モデルを利用することで計算速度の向上を図りたい.
   
  \subsection{相対速度}
  ダッシュポッドが粒子に与える力を計算するには粒子間の相対速度が必要である.\\
  法線方向の相対速度は, 粒子の相対速度の法線方向の成分を取り出せば良い.
  \begin{flalign} 
   {\boldsymbol {\boldsymbol v} }_{n_{ij}} &= [({\boldsymbol v}_i - {\boldsymbol v}_j) \cdot {\boldsymbol n}_{C_{ij}}]  {\boldsymbol n}_{C_{ij}}
  \end{flalign}


 接線方向の相対速度は, 相対速度から法線成分を抜いて, 回転運動成分を足したものである.
 \begin{flalign}
  {\boldsymbol v}_{t_{ij}} = {\boldsymbol v}_{ij} 
  -  ( {\boldsymbol v}_{ij} \cdot {\boldsymbol n}_{C_{ij}} ) {\boldsymbol n}_{C_{ij}} \nonumber \\
  + (r_i{\boldsymbol \omega}_i   + r_j{\boldsymbol \omega}_j)\times {\boldsymbol n}_{C_{ij}} 
 \end{flalign}
 ここで, $r_i$は粒子$i$の半径であり, $\omega_i$は粒子$i$の角速度である.


 \subsection{衝突点の算出}
 球体の場合, 対称性を考慮すると, 以下の式で衝突点が与えられる.
 \begin{flalign}
  \delta_{ij} = |{\boldsymbol x_i} - {\boldsymbol x}_j| - (r_i + r_j)\\
  {\boldsymbol n}_{C_{ij}} = ({\boldsymbol x}_j -{\boldsymbol x}_i)_{normalize}
 \end{flalign}
 ここで, $x_i$は粒子$i$の中心の座標である.\\
 球体以外を利用する場合, 粒子がオーバーラップしている領域の接戦を利用することで衝突点が検出可能なアルゴリズムが提案されている\cite{elipsoid_collision}.




\section{吸着力}
ファンデルワールス力による粒子間吸着が生じる.
ハマーカー数$H_A$, 換算粒子径$d^*$, 表面間距離$h$, 法線${\boldsymbol n}$を用いて

\begin{flalign}
 {\boldsymbol F} = \frac{H_A d^*}{24h^2}{\boldsymbol n}
\end{flalign}
のように与えられる.
換算粒径$d^*$は, 粒子同士の衝突において
\begin{flalign}
 d^* = \frac{d_i d_j}{di + dj}
\end{flalign}

粒子と壁との衝突では
\begin{flalign}
 d^* = di
\end{flalign}
のように与えられる.表面間距離$h$が0になると計算が発散してしまうので, 0.4nm(経験則)をカットオフ値とした切り捨て処理をする.
また, ファンデルワールス力が働く範囲は, 実装(空間分割の解像度など)に依存する.\\
粘性と異なり, ファンデルワールス力は接触していない粒子間にも働く力である.
粘性だけしか考慮しない場合, 粒子がダイナミックな動きをすると, 衝突が起きずらく, 実物よりも流動性が高くなってしまう.\\
しかし, 粉体の粒子間に働くファンデルワールス力についてはまだわからないことが多く, ホットな研究分野になっている.


\section{粉体崩壊曲線}
粉体内部は粒子同士の相互作用が大きく, ある条件を満たさなければ動き出すことができない.
すなわち, 式\eqref{eq:base_equation}の右辺がある程度の大きさの力でなければ
その条件を数式化したのが粉体崩壊曲線である.\\
粉体の内部の微小面に注目する.
その面を全方位に回転させて, それぞれの方向の垂直応力と剪断応力を測定す.
測定した垂直応力の中で, 最大のものと最小のものをそれぞれ$\sigma_{max}$, $\sigma_{min}$とすると, 以下の関係式が導かれる.
\begin{flalign}
\label{eq:mohr_stress_circle}
 \left(\sigma - \frac{\sigma_{max} + \sigma_{min}}{2} \right)^2 + \tau^2 &= \left(\frac{\sigma_{max} - \sigma_{min}}{2} \right)^2
\end{flalign}

$\sigma$は垂直応力, $\tau$は剪断応力である.
$\sigma-\tau$平面において, この式は円の形を意味する.
この円をモール(Mohr)応力円とよぶ.
式\eqref{eq:mohr_stress_circle}より, 垂直応力の最大値と最小値がわかればこの円を定義することができる.\\
%また, 垂直応力の最大値と最小値はある範囲でしかその値をとりえないことがわかっている.
モール応力円を複数書いていくと, ある包絡線が描ける.
それが粉体崩壊曲線である.

\begin{flalign}
 \label{eq:warren_spring}
 \left(\frac{\tau}{\tau_c} \right)^n &= \frac{\sigma + \sigma_t}{\sigma_t}
\end{flalign}
式\eqref{eq:warren_spring}をワーレンスプリング(Warren--Spring)式\cite{warren_spring}とよぶ.
$\tau_c$は粘着力を意味し, 大きいほど粉体は崩壊しにくくなる. 
$\sigma_t$は引っ張り強度を意味し, 垂直応力を大きくしたときの崩壊のしやすさを表す.
左辺の次数は剪断指数とよばれ, この値が大きいほど粉体崩壊曲線の曲率が大きくなり, 粉体の付着性が強いことを意味する.\\
この2つの式が交点を持つとき, 粉体層は動き出す, すなわち崩壊する.

\subsection{崩壊曲線のための離散化}
モールの応力円を定義するためには, 各粒子のすべての面に対してかかる力を求めなければならない.
粉体を離散化するために球体を利用しているが, 球体は無数に面が存在する多面体と考えられるので, 粒子自体を離散化する必要がある.\\
各面にかかる力を測定する必要があるので, 粒子を離散化する図形はすべての面が合同な図形である正多面体がいいと考えられる.
しかし, 離散化に使用する正多面体の麺の数が多すぎても見た目には影響してこないかもしれず, 少なすぎると形状が球体から遠ざかってしまう.
以上のことから, 以下の離散化図形が考えられる.
\begin{itemize}
 \item 正四面体
 \item 正六面体(立方体)←最有力
 \item 正八面体
\end{itemize}

$n$面体として離散化できれば, $n$面にかかる力をそれぞれ計算できるので, その中の最大値と最小値を用いて描いたモール応力円で粉体曲線との交差を判定することできる.
いろいろな形状で試してみるとよかろう.\\
球体を多面体で離散化するとき, 面の数が多いほど元の形状である球体に近くが, 計算すべき面の数は増える.
いっぽうで, 面の数が少なければ計算すべき面の数は少なくて済むが, その形状は球体から遠ざかる.
このトレードオフが見た目にどの程度影響してくるかは実際にシミュレーションしてみないとわからない.\\
粒子の離散化形状は, 球体に対して内接するか概説するかでも結果が異なることが考えられる.
外接する形状と内接する形状の2つを考えて, その平均を取ってみても結果が変わるかもしれない.


\section{TODO List}
\begin{enumerate}
 \item 死ぬ気で実装すること
 \item 崩壊判定と形状の関係性における調査
 \item 実物との比較実験
 \item 既存研究との対比
\end{enumerate}


%%%%%%%%%%%%% 参考文献 %%%%%%%%%%%

%% スタイルファイル
%\bibliographystyle{plain}%標準
%\bibliographystyle{abbrv}%author, editorのファーストネームが省略される
%\bibliographystyle{alpha}%author, year は省略形
%\bibliographystyle{acm}%ACMのやつ
\bibliographystyle{ieeetr}%IEEEのやつ
%\bibliographystyle{unsrt}%plainと同じ, 文献が引用順に並ぶ

%% bibファイル読込み
\bibliography{reference}

\end{document}

